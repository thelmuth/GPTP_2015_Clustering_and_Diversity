%%%%%%%%%%%%%%%%%%%% author.tex %%%%%%%%%%%%%%%%%%%%%%%%%%%%%%%%%%%
%
%%%%%%%%%%%%%%%% Springer %%%%%%%%%%%%%%%%%%%%%%%%%%%%%%%%%%

\title*{Diversity and Clustering}
% Use \titlerunning{Short Title} for an abbreviated version of
% your contribution title if the original one is too long
\author{Tom}
% Use \authorrunning{Short Title} for an abbreviated version of
% your contribution title if the original one is too long
\institute{Name of First Author \at Name, Address of Institute
\and Name of Second Author \at Name, Address of Institute}

\maketitle

\abstract{Each chapter should be preceded by an abstract (10--15 lines long) that summarizes the content. The abstract will appear \textit{online} at \url{www.SpringerLink.com} and be available with unrestricted access. This allows unregistered users to read the abstract as a teaser for the complete chapter. As a general rule the abstracts will not appear in the printed version of your book unless it is the style of your particular book or that of the series to which your book belongs.}

\begin{keywords}
keywords to your chapter, these words should also be indexed
\end{keywords}
\index{keywords to your chapter}
\index{these words should also be indexed}
\\



\section{Section Heading}
\label{sec:1}

Lexicase citation \citep{Helmuth:2015:ieeeTEC}.



\begin{acknowledgement}
Later.
\end{acknowledgement}

\bibliographystyle{spbasic}
\bibliography{gp-bibliography,spector}
